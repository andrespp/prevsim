\begin{frame}{Ofensores}
  \begin{block}{$P()$}
    \begin{itemize}
      \item Dados agregados no AEPS
      \item Projeção do IBGE não distingue clientela
      \item Dificuldade no cálculo de $Ps()$
    \end{itemize}
  \end{block}
  \begin{block}{$Eb()$ para todos os anos da série histórica}
    \begin{itemize}
      \item Microdados possuem estoque em 2016
      \begin{itemize}
        \item Microdados não possuem a data de cessação de benefícios cessados
        \item Não é possível o cálculo do estoque em anos anteriores
      \end{itemize}
    \end{itemize}
  \end{block}
\end{frame}

\begin{frame}
  \begin{block}{Objetivos}
    \begin{itemize}
      \item Reproduzir o modelo de projeção oficial utilizando os microdados
      \begin{itemize}
        \item Ratificar o grau de incerteza obtido nos dados agregados
        \item \alert{Novas propostas de reforma}
      \end{itemize}
      \item \alert{Simular a reforma da previdência com os dados históricos}
      \item Propor um novo modelo de projeção de longo prazo para a previdência
      social no Brasil
    \end{itemize}
  \end{block}
\end{frame}

\begin{frame}{Próximos passos}
  \begin{block}{$P()$}
    \begin{itemize}
      \item Projeção da população brasileira separada por:
      \begin{itemize}
         \item Ano (2010 a 2060)
         \item Idade
         \item Sexo
         \item Clientela (Rural/Urbana)
      \end{itemize}
    \end{itemize}
  \end{block}
  \begin{block}{Nova proposta de reforma}
    \begin{itemize}
      \item Maximar estoque de contribuintes
      \scriptsize
          \begin{center}
          $\sum_i \sum_s \sum_c C(i,t,s,c) =$ \\
           $\sum_i \sum_s \sum_c P(i,t,s,c) \times Part(i,t,s,c) \times [1-Desemp(i,t,s,c)] \times d(i,t,s,c)$
          \end{center}
     \item Variando:
      \begin{itemize}
          \scriptsize
          \item $Part$   %é a taxa de participação na força de trabalho
          \item $Desemp$ %é a taxa de desemprego
          \item $d$ %é a densidade da contribuição (proporção de meses de
            %contribuição do empregado no ano, onde $d=1$ significa 12 meses de
            %contribuição
      \end{itemize}
    \end{itemize}
  \end{block}
\end{frame}

%\begin{frame}
%  \begin{block}{}
%  \end{block}
%\end{frame}

%\begin{frame}
%  \begin{figure}[h]
%  	\begin{center}
%      \includegraphics [scale=0.3]{./Figures/Device-Estimates}
%     % \caption {Estimativa de dispositivos conectados à Internet.}
%  		%\label{fig:arq-imuno}
%  	\end{center}
%  \end{figure}
%\end{frame}

%\begin{frame}{Redes de Acesso}
%	\begin{figure}[!htb]
%		\centering
%		\subfloat[DSL]{
%			\includegraphics[height=3.5cm]{./Figures/DSLaccess}
%			\label{figdroopy}}
%		\quad %espaco separador
%		\subfloat[Cable]{
%			\includegraphics[height=3.5cm]{./Figures/CableAccess}
%			\label{figsnoop}}
%		%\caption{Subfiguras}
%		%\label{fig01}
%	\end{figure}
%\end{frame}

%\begin{frame}[fragile]
%\scriptsize
%\begin{verbatim}
%\end{verbatim}
%\end{frame}

%\begin{frame}{\textit{Socket Programming with TCP}}
%\scriptsize
%\lstinputlisting[language=Python, caption={TCP Server.}]{./code/upperServer/TCPserver.py}
%\end{frame}

