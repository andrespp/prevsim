\begin{frame}
  \begin{block}{Instituto Nacional do Seguro Social - INSS}
    \begin{itemize}
      \item O INSS é uma autarquia responsável pela implantação da previdência
      social no Brasil
      \item Escopo:
      \begin{itemize}
        \item Reconhecer e operacionalizar os direitos dos segurados do Regime
        Geral de Previdência Social (RGPS), a previdência pública brasileira.
      \end{itemize}
    \end{itemize}
  \end{block}
  \begin{block}{RGPS}
    \begin{itemize}
      \item Possui caráter contributivo e de filiação \alert{obrigatória}
      \begin{itemize}
        \item Fator relevante no planejamento individual dos cidadãos
        \item Peça central de atenção no planejamento público
      \end{itemize}
    \end{itemize}
  \end{block}
\end{frame}

\begin{frame}
  \begin{block}{Sistema Insolvente?}
    \begin{itemize}
      \item Preocupação global com a solvência dos regimes de previdência
      pública
      \item Brasil vive um momento onde discute-se \alert{profundas mudanças}
    \end{itemize}
  \end{block}
  \pause
  \begin{block}{Mudanças Propostas}
    \begin{itemize}
      \item Critérios propostos amparados em estudos estatísticos
      \item Contudo, pesquisas independentes divergem das projeções oficiais
      \cite{reformarparaexcluir}.
    \end{itemize}
  \end{block}
\end{frame}

\begin{frame}
  \begin{block}{Benefícios Previdenciários}
    \begin{itemize}
      \item Prestações pecuniárias pagas pela Previdência Social aos segurados
      ou aos seus dependentes de forma a atender a cobertura dos
      \alert{eventos}
      \begin{itemize}
        \item Doença, invalidez, morte, idade avançada...
      \end{itemize}
      \item \textbf{Benefícios de Prestação Continuada}
      \begin{itemize}
        \item Aposentadorias, Pensões por Morte, Auxílios, rendas mensais
        vitalícias, abonos por permanencia em serviço, salários família e
        maternidade
      \end{itemize}
      \item \textbf{Benefícios de Prestação Única}
      \begin{itemize}
        \item Pecúlio especial de aposentados
        \begin{itemize}
          \item Extinto pela lei 8.870/94 mas ainda pago para alguns
          contribuintes
        \end{itemize}
      \end{itemize}
    \end{itemize}
  \end{block}
\end{frame}

%\begin{frame}
%  \begin{block}{}
%    \begin{itemize}
%      \item
%    \end{itemize}
%  \end{block}
%\end{frame}

%\begin{frame}
%  \begin{block}{}
%  \end{block}
%\end{frame}

%\begin{frame}
%  \begin{figure}[h]
%  	\begin{center}
%      \includegraphics [scale=0.3]{./Figures/Device-Estimates}
%     % \caption {Estimativa de dispositivos conectados à Internet.}
%  		%\label{fig:arq-imuno}
%  	\end{center}
%  \end{figure}
%\end{frame}

%\begin{frame}{Redes de Acesso}
%	\begin{figure}[!htb]
%		\centering
%		\subfloat[DSL]{
%			\includegraphics[height=3.5cm]{./Figures/DSLaccess}
%			\label{figdroopy}}
%		\quad %espaco separador
%		\subfloat[Cable]{
%			\includegraphics[height=3.5cm]{./Figures/CableAccess}
%			\label{figsnoop}}
%		%\caption{Subfiguras}
%		%\label{fig01}
%	\end{figure}
%\end{frame}

%\begin{frame}[fragile]
%\scriptsize
%\begin{verbatim}
%\end{verbatim}
%\end{frame}

%\begin{frame}{\textit{Socket Programming with TCP}}
%\scriptsize
%\lstinputlisting[language=Python, caption={TCP Server.}]{./code/upperServer/TCPserver.py}
%\end{frame}
