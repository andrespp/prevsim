\begin{frame}
  \begin{block}{}
    \begin{itemize}
      \item Em 2017, Carlos Patrick \cite{patrick} demonstrou que os dados
      oficiais do governo referentes à previdência social apresentam
      \textbf{projeções} \alert{viesadas no curto prazo}, e com \alert{erros
elevados no longo prazo}.
      \item Alertou ainda para a \alert{impossibilidade de reprodução dos
resultados} das Leis de Diretrizes orçamentárias (LDOs) de 2012 e 2018
      \begin{itemize}
        \item Falta de tansparência nos dados oficiais
        \item Falta de tansparência na metodologia
      \end{itemize}
    \end{itemize}
  \end{block}
\end{frame}

\begin{frame}
  \begin{block}{CPI da Previdência}
    \begin{itemize}
      \item Após a publicação da tese do Prof. Carlos Patrick, foram
      disponibilizados pelo governo os microdados (dados não agregados) da base
      utilizada naquela pesquisa
      \item Torna-se possível avaliar as variáveis que sofrerão mudanças com a
      reforma:
      \begin{itemize}
        \item  Melhores análise
        \item  Melhores projeções
        \item  Exercício de novos modelos
      \end{itemize}
    \end{itemize}
  \end{block}
\end{frame}

\begin{frame}
  \begin{block}{Objetivos}
    \begin{itemize}
      \item Reproduzir o modelo de projeção oficial utilizando os microdados
      \begin{itemize}
        \item Ratificar o grau de incerteza obtido nos dados agregados
        \item \alert{Novas propostas de reforma}
      \end{itemize}
      \item Simular a reforma da previdência com os dados históricos
      \item Propor um novo modelo de projeção de longo prazo para a previdência
      social no Brasil
    \end{itemize}
  \end{block}
\end{frame}

%\begin{frame}
%  \begin{block}{}
%    \begin{itemize}
%      \item
%    \end{itemize}
%  \end{block}
%\end{frame}

%\begin{frame}
%  \begin{block}{}
%  \end{block}
%\end{frame}

%\begin{frame}
%  \begin{figure}[h]
%  	\begin{center}
%      \includegraphics [scale=0.3]{./Figures/Device-Estimates}
%     % \caption {Estimativa de dispositivos conectados à Internet.}
%  		%\label{fig:arq-imuno}
%  	\end{center}
%  \end{figure}
%\end{frame}

%\begin{frame}{Redes de Acesso}
%	\begin{figure}[!htb]
%		\centering
%		\subfloat[DSL]{
%			\includegraphics[height=3.5cm]{./Figures/DSLaccess}
%			\label{figdroopy}}
%		\quad %espaco separador
%		\subfloat[Cable]{
%			\includegraphics[height=3.5cm]{./Figures/CableAccess}
%			\label{figsnoop}}
%		%\caption{Subfiguras}
%		%\label{fig01}
%	\end{figure}
%\end{frame}

%\begin{frame}[fragile]
%\scriptsize
%\begin{verbatim}
%\end{verbatim}
%\end{frame}

%\begin{frame}{\textit{Socket Programming with TCP}}
%\scriptsize
%\lstinputlisting[language=Python, caption={TCP Server.}]{./code/upperServer/TCPserver.py}
%\end{frame}


